\documentclass{article}

% \usepackage[spanish]{babel}

\usepackage[utf8]{inputenc}
\usepackage{graphicx}
\usepackage{amsfonts}
\usepackage{amsmath}

\title{Ecuaciones en diferencias}

\author{Alumnos de 3er semestre grupo 2}

\date{18 de septiembre de 2017}

\begin{document}

\maketitle

Si en una ecuación en diferencias, la función f no depende de n, la ecuación en diferencias es autónoma. $\mathbb{R}$.

\section{Ecuaciones de primer orden}

\subsection{Ecuaciones lineales}

Una ecuación lineal en diferencias de primer orden tiene la forma $x_{n+1}=ax_n$ donde $a$ es una constante. 

La fórmula para resolver ecuaciones lineales es:
\begin{equation}
  \label{lineal}
  x_n=a^nx_0
\end{equation}

Por ejemplo, si inciamos una inversión con 1000 pesos con un interés mensual del 1\%, obtenemos lo siguiente:

\begin{center}
  \includegraphics[width=8cm]{inversion.png}
\end{center}


$x_{n+1}=1.01x_t$

Estas ecuaciones hacen que una ecuación que inicialmente era de esta manera $x_{n+1}=x_n+2(x_n)$ puede escribirse de manera que la ecuación solo dependa de $x_n$ y queda de la siguiente forma:

$x_n=3^n(4)$0     

\subsection{Ejemplito}

Nuestro compañerito Pepe muere en raras circunstancias (después de tener su examen de Álgebra II), luego descubren lo radiante de su ser. cada 20 años el elemento radioactivo decae a razón de $2\%$, si inicialmente contenía 2 kilitos de puro amor ¿Cuánto tendrá cuando se vea un poco más feo (como quien digo, dentro de 200 años)?

\subsection{Un ejemplito más}

(esto continuará, no lo pude hacer completo)

\section{Ecuaciones de segundo orden}

El método para resolver estas ecuaciones está inspirado en la fórmula \ref{lineal}.

Para resolver una ecuación en diferencias de segundo orden se usa la ecuación resolvente.

<<<<<<< HEAD
\subsection{Ejemplo no.1}

La ecuación de segundo grado mas conocida es la $Sucesión de Fibonacci$ la cual se ve de la forma:

$$a_{k+2}=a_{kn+1}+a_{kn}$$
 
En donde $a_{0}=1$ y $a_{1}=1$

Claramente es facil calcular los primeros terminos, pero, ¿Como calcular la cuando el termino es demasiado grande?
La respuesta es sencilla, resolvemos una ecuacion resovente que viene de la solucion, la cual se ve de la siguiente manera:

$$x^2-x-1$$

Obtenemos despues ambas raices, las cuales son:
$x_{1}= {1+(5)^{1/2}}/2$ y $x_{2}={1-(5)^{1/2}/2$

Obteniendo finalmente:
$alpha_{1}(1+(5)^{1/2})^n + alpha_{2}(1-(5)^{1/2}/2)^n$

Ahora resolvemos un sistema de ecuaciones tal que:
$alpha_{1} + alpha_{2}= 1$ y $alpha_{1}(1+(5)^{1/2}) + alpha_{2}({1-(5)^{1/2}/2)=1$

Para obtener finalmente:
$F_{n}= (5+(5)^{1/2})/10)((1+(5)^{1/2})^n)/2) + (5-(5)^{1/2})/10)((1-(5)^{1/2})^n)/2)$

\subsection{Ejemplo no.2}
=======
Una ecuación en diferencias de segundo orden , tiene la forma $a_{k+2}=f(a_k ,a_{k+2}$

Esta ecuación no tiene raices reales $$x^2+1=0$$

\subsection{Ejemplo}

¿De cuantas maneras se puede cubrir un tablero de $2xn$ usando fichas de $1x2,2x1$?
>>>>>>> devel

En la siguiente ecuación en diferencias
\begin{equation}
  \label{eq:1}
  x_{n+2}=4x_{n+1}-4x_{n}
\end{equation}

veremos que al resolverla tenemos una ecuación con una sola raiz.
En $$r^2-4r+4=0$$ (2) las raices son soluciones de (1), usado la formula chicharronera vemos que solo con $r=2$.

La ecuaci\'on resolvente es:
<<<<<<< HEAD
$$x_n=\lamda_1(2^n)+\lamda_2(n)(2^n)$$
ya solo despues se checan los valores iniciales para que nos de los valores de $$\lamda_1$$ y $$\lamda_2$$.

=======

$$x_n=m_1(2^n)+m_2(n)(2^n)$$
ya solo despues se checan los valores iniciales para que nos de los valores de $$m_1$$ y $$m_2$$.

$$x_n=m_1(2^n)+m_2(n)(2^n)$$
ya solo despues se checan los valores iniciales para que nos de los valores de $m_1$ y $m_2$.
>>>>>>> devel

\subsection{Números  complejos}

Una ecuación en diferencias de segundo orden , tiene la forma $a_{k+2}=f(a_k ,a_{k+2}$
Esta ecuaci\'on no tiene raices reales $$x^2+1=0$$

Un complejo es un número de la forma $z=a+bi$, $a,b\in\mathbb{R}$ , con $i^2=(-1)$, a parte real, b parte imaginaria.

El módulo de un número complejo: $|a+bi|=\surd(a^2+b^2)$.


El argumento de un número complejo es el ángulo comprendido en(continuará)


\subsection{Ejemplo}

Para cada $n$, considera el determinante $n\times n$ dado por:
\begin{equation}
  \label{eq:1}
  D_n=\det
  \begin{pmatrix}
    b & b\\
    b & b 
  \end{pmatrix}
\end{equation}

\subsection{RAICES COMPLEJAS}

Algunas ocaciones las ecuaciones de segundo orden no tendran solución en los reales, sin embargo en los complejos si y no por manejar numeros complejos su nivel de dificultad sera mayor. Aqui mostramos un ejemplo de soluciones complejas:

$x_{n+2}=-4x_{n}$ con $x_0=1$ y $x_1=0$

Resolvemos igual que en ecuaciones de segundo orden entonces aplicamos la formula general para buscar sus raices y tenemos:

$r=\pm2i$

Ahora buscaremos su módulo:

$|r|=\sqrt{2^2}$

Este ejemplo continuará...

\end{document}
